\section{Excellence}
\note{Should start on page 5; aim for 10 to 12 pages.}

\subsection{Quality, innovative aspects and credibility of the research
  programme}
\note{including inter/multidisciplinary, intersectoral and, where appropriate,
  gender aspects}

\subsubsection{Introduction, objectives and overview of the research programme}
\note{For ETN, it should be explained how the individual projects of the recruited researchers will be integrated into --- and contribute to --- the overall research programme. EJD and EID proposals should describe the research projects in the context of a doctoral training programme}

\itncaption{Table 1.1: Work Package (WP) List}
\note{The WP names are defined in common.tex}
\begin{itntable}{|>{\ra}p{10mm}|>{\ra}p{35mm}|>{\ra}p{15mm}|>{\ra}p{10mm}|>{\ra}p{10mm}|>{\ra}p{20mm}|>{\ra}p{20mm}|>{\ra}p{25mm}|}
    \colorrow
    \hline
    \textbf{WP No.} &
    \textbf{WP Title} &
    \textbf{Lead Beneficiary No.} &
    \textbf{Start Month} &
    \textbf{End Month} &
    \textbf{Activity Type} &
    \textbf{Lead Beneficiary Short Name} &
    \textbf{ESR Involvement} \\
    \hline
    &&&&&&& \\
    \hline
\end{itntable}

\subsubsection{Research methodology and approach}
\label{sec:method}

Here you can cite with footnotes\cite{Bayes1763}.

\subsubsection{Originality and innovative aspects of the research programme}

\itnsection{Beyond the state of the art}

\itnsection{Existing programmes}

\subsection{Quality and innovative aspects of the training programme}
\note{including transferable skills, inter/multi-disciplinary, inter-sectoral and, where appropriate, gender aspects}

\subsubsection{Overview and content structure of the training}
\label{sec:training}
\note{including network-wide training events and complementarity with those programmes offered locally at the participating organisations (please include table 1.2a and table 1.2b)}

\itncaption{Table 1.2a: Recruitment Deliverables per Beneficiary}
\begin{itntable}{|>{\ra}p{30mm}|>{\ra}p{30mm}|>{\ra}p{30mm}|>{\ra}p{30mm}|>{\ra}p{30mm}|}
    \colorrow
    \hline
    \textbf{Researcher No.} &
    \textbf{Recruiting Participant (short name)} &
    \textbf{PhD awarding entities} &
    \textbf{Planned Start Month 0--45} &
    \textbf{Duration (months) 3--36} \\
    \hline
    &&&& \\
    \hline
    Total & & & & XXX \\
    \hline
\end{itntable}

\itncaption{Table 1.2b: Main Network-Wide Training Events, Conferences and Contribution of Beneficiaries}
\begin{itntable}{|>{\ra}p{10mm}|>{\ra}p{85mm}|>{\ra}p{20mm}|>{\ra}p{20mm}|>{\ra}p{20mm}|}
    \colorrow
    \hline &
    \textbf{Main Training Events \& Conferences} &
    \textbf{ECTS (if any)} &
    \textbf{Lead Institution} &
    \textbf{Action Month (estimated)} \\
    \hline
    &&&& \\
    \hline
\end{itntable}


\subsubsection{Role of non-academic sector in the training programme}

\subsection{Quality of the supervision}
\note{including mandatory joint supervision for EID and EJD}

\subsubsection{Qualifications and supervision experience of supervisors}

\subsubsection{Quality of the joint supervision arrangements}
\note{mandatory for EID and EJD}
\note{Would make sense here to emphasise the multidisciplinarity - add a table?}


\subsection{Quality of the proposed interaction between the participating organisations}

\subsubsection{Contribution of all participating organisations to the research and training programme}

\subsubsection{Synergies between participating organisations}
\label{sec:synergies}


\subsubsection{Exposure of recruited researchers to different (research)
  environments, and the complementarity thereof}


%%% Local Variables:
%%% mode: latex
%%% TeX-master: "master"
%%% TeX-PDF-mode: t
%%% End:
